\documentclass[12pt]{article}

\usepackage{sbc-template}

\usepackage{graphicx,url}

%\usepackage[brazil]{babel}   
\usepackage[latin1]{inputenc}  

     
\sloppy

\title{Trabalho Rápido 3\\Metas de Usabilidade}

\author{Brunno Manduca do Prado}


\address{Universidade do Estado de Santa Catarina - UDESC\\
 \email{brunnomanduca@outlook.com}
	\vspace{1cm}
}

\begin{document} 

\maketitle

\begin{abstract}

\end{abstract}
     
\begin{resumo} 

\end{resumo}


\section{Metas de Usabilidade - Segundo Diversos Autores\\}

\subsection{Visão Nielsen(1993)}

		Nielsen(1993):Segundo Nielsen (1993) usabilidade é um conceito que busca definir as características de 
	utilização, do desempenho e da satisfação dos usuários na interação com as interfaces 
	computacionais, na perspectiva de um bom sistema interativo, e se refere a cinco componentes: 

	Aprendizado: O sistema em questao deve ser de facil aprendizado. Utilizadores e/ou usuarios inexperientes
	devem se esforçar ou ser capazes de realizar tarefas basicas em um curto espaço de tempo e com minimo de informaao possivel.

	Eficiencia: Usuarios mais avançados devem se esforçar ou ser capazes de alcançar um estado avançado de produtividade.

	Memoravel: Usuarios podem ter acesso ao o sistema, mesmo depois de um grande periodo sem uso e ainda completar as tarefas,
	sem necessidade de realziar treinos novamente e tambem deve ser de facil recordaçao.

	Baixo numero de Erros: Todos os usuarios precisam enfrentar a quantidade menos possivel de erros durante o uso do sistema e 
	poder ter recuperaçao de forma rapida do mesmo.

	Satisfação dos usuarios: O sistema precisar agradar a todo e qualquer usuario.

\subsection{Visão Preece et al. (2005)}

		Preece(2005): Segundo Preece,é importante que exista um equilíbrio entre as metas de usabilidade e as decorrentes da
	experiência do usuário. Tullis et al. (2008) afirmam que a experiência do usuário faz referência a todos aspectos da interação entre um determinado sujeito e um produto, uma aplicação ou sistema. Desse modo as métricas de usabilidade podem revelar 			algo sobre a experiência do usuário, como aspectos relacionados à :

		Eficácia: Um sistema só é útil se seus usuários são capazes de atingir os objetivos pretendidos. Um sistema ineficaz pode ser deixado de lado. A eficácia é medida através da capacidade dos usuários em completar uma tarefa em particular ou não. 			Esta abordagem é adequada para a maioria dos estudos em que a tarefa consiste de um único passo ou caminho. No entanto, tarefas complexas podem exigir uma definição mais detalhada de sucesso ou fracasso, podendo incluir níveis tais como a falha 			ou sucesso parcial.
		
		Satisfação: enquanto análises objetivas de usabilidade de sistemas são comuns, a avaliação subjetiva dos usuários é fundamental para o sucesso de um sistema. Um sistema é fadado a falhar, mesmo quando tem boa usabilidade, 
	se não for popular entre os usuários. A satisfação do usuário pode ser avaliada através de entrevistas e escalas.

		Precisão: o fator de precisão foi identificado principalmente em tarefas de autenticação. Em muitos casos, sistemas de autenticação exigem que os usuários digitem a senha com 100% de precisão. Estas exigências podem ser impactadas por outras 		demandas ambientais, como a memorização de informação ou fatores pessoais.

		Eficiência: Enquanto os usuários utilizam um sistema para atingir um determinado objetivo, atingir este, por si só, não é suficiente. O objetivo deve ser alcançado dentro de níveis de tempo e esforço aceitáveis. O nível aceitável de tempo ou esforço 		no contexto de um sistema pode não o ser em outro. A eficiência é capturada através 	da medição de tempo para completar uma tarefa ou o número de cliques/ botões pressionados para atingir as metas exigidas.

		Memorização: muitos sistemas de autenticação exigem que os usuários memorizem segredos necessários para adentrar ao sistema. O número de segredos que um usuário é obrigado a guardar aumenta com o número de sistemas de autenticação 				diferentes que ele interage. Isso resulta em problemas de memorização, onde os usuários têm dificuldades para se autenticar em vários sistemas diferentes, muitas vezes requisitando opções de refazer as senhas.

		Habilidade: Isto é baseado na suposição de que os usuários vão aprender ou realmente tentar aprender e entender o sistema. Este pressuposto é falho particularmente em sistemas seguros. Usuários só se preocupam com as partes que eles acham 	que são importantes para as operações específicas que necessitam fazer, e em muitos casos as tarefas de segurança não são vistas como importantes. Estudos anteriores também descobriram que treinar os usuários na utilização de sistemas de segurança 	é ineficaz.


\subsection{Visão Norma ISO 9241}
	
		A parte 11 (1998) desta norma redefine usabilidade como "a capacidade de um produto ser usado por usuários específicos para atingir objetivos 
		específicos com eficácia, eficiência e satisfação em um contexto específico de uso."

	Para melhor compreensao foi esclarecido outros conceitos:

	Usuario: Pessoa que ira interagir com todo o sistema/produto

	Constexto de uso: Se baseia nos usuarios, nas tarefas e equipamentos(Hardware,Software,materiais), ambiente fisico e social em que o produto sera usado.

	Eficacia: Parte para precisao e completeza ao qual os usuarios atingem os objetivos de alguma formaespecificos, acessando ou buscando a informaçao 
	mais correta ou gerando os resultados esperado

	Eficiencia:Parte de precisao e completeza com parte dos usuarios que atingem seus objetivos, em relaçao a toda a quantidade de recursos gastos

	Satisfaçao: Parte de conforto e aceitabilidade do produto, medido atraves ou por meio de metodos subjetivos e/ou objetivos.

	


\section{Exemplos de Utilização de Cada Meta}


\section{Análise Comparativa}


%\section{First Page} \label{sec:firstpage}


%\section{Figures and Captions}\label{sec:figs}


\section{Images}

\begin{figure}[ht]
\centering
\includegraphics[width=.7\textwidth]{Fig1.jpeg}
\caption{Diferença entre a visão de Nielsen e a Norma ISO 9126}
\end{figure}

\begin{figure}[ht]
\centering

\end{figure}



\begin{table}[ht]
\centering

\end{table}





\section{References}

Bibliographic references must be unambiguous and uniform.  We recommend giving
the author names references in brackets, e.g. \cite{knuth:84},
\cite{boulic:91}, and \cite{smith:99}.

The references must be listed using 12 point font size, with 6 points of space
before each reference. The first line of each reference should not be
indented, while the subsequent should be indented by 0.5 cm.

\bibliographystyle{sbc}
\bibliography{sbc-template}

\end{document}
